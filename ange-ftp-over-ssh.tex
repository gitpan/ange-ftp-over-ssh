\documentstyle[html]{article}
\title{Ange ftp over ssh}
\author{Russell Standish\\http://parallel.hpc.unsw.edu.au/rks}
\begin{document}
\maketitle

\begin{abstract}
GNU emacs has a wonderful remote file editing facility called Ange
ftp. However, because it uses ftp as its file transport agent,
passwords are transmitted as plain text which can be snooped by the
unscrupulous ``bad guys'' out there in cyberspace. This package is a
``drop in replacement'' for the ftp client that instead redirects the
file transfers over ssh, which allows for connections without the ned
for plain text passwords to be transmitted.
\end{abstract}

\section{Prerequisites}

You should have \htmladdnormallink{GNU
  Emacs}{http://www.gnu.org/software/emacs/emacs.html} and
\htmladdnormallink{ssh}{http://www.ssh.org/} installed, which you
probably have already. You will also need \htmladdnormallink{Perl
  5}{http://www.perl.com} installed. Then you need to download the
\htmladdnormallink{nftp}{ftp://parallel.hpc.unsw.edu.au/software/ange-ftp-over-ssh.tar.gz}
client software, written in perl. This software has only been tested
on Unix platforms, but may well run on Windows without much change.

If you are not already familiar with ange ftp, you should read the
\htmladdnormallink{Emacs
  manual}{http://www.gnu.org/manual/emacs/html_node/emacs_150.html#SEC156}.
In brief, you can open remote files for editing using the syntax:
\begin{verbatim}
/host:filename
/user@host:filename
\end{verbatim}

\section{Getting ssh to connect without prompting for a password}

You also need to get ssh to connect to your remote site without
prompting for a password, as ssh reads the keyboard directly, and
cannot be fed a password directly from {\tt nftp.pl}. There are two
ways to do this:
\begin{enumerate}
\item Set up a .rhosts or .shosts file on your remote system with you
  local hostname and userid. This technique is simple, but requires
  the local workstation to have a well known address (not true of ISP
  dial account, for example). It is also susceptible to IP spoofing
  attacks, and may well have been disabled by the system administrator
  of the remote site for this reason.
\item Set up an {\tt ssh-agent} to supply your bona fides to the
  remote system using RSA public key encryption. My setup has the
  following lines in my {\tt .profile}:
\begin{verbatim}
ttname=`tty`
if [ "${ttname%%[0-9]}" = "/dev/tty" ]; then
  eval `ssh-agent`
  ssh-add
fi
\end{verbatim}
This script (which you may need to modify for non-Linux OSes) will set
up an {\tt ssh-agent}, and prompt you for your password to load your
private key into the agent's database.

You now need to copy your public key (located in
\verb+.ssh/identity.pub+) into the files \verb+~/.ssh/authorized_keys+
and \verb+~/.ssh/known_hosts+ on the remote remote system. This should
enable the remote system to authenticate your ssh connection, using
the public key information supplied by {\tt ssh-agent}.
\end{enumerate}

\section{Configuration file \~{}/.nftprc}

Create the file \verb+~/.nftprc+ containing a list of machines you
wish to remotely edit via ssh with the following sample format:
\begin{verbatim}
$aliases{grimble}="grimble.north-pole.com:22";
$aliases{grunge}="localhost:2000";
\end{verbatim}
Any machine name not mentioned in this file will be connected to by
the usual ftp method. In the above example, two hostnames are defined,
grimble, and grunge. In the first case, {\tt nftp.pl} will ssh to grimble
on port 22 (the standard ssh port). In the second case, the standard
ssh port of grunge has been forwarded to port 2000 on localhost (by another ssh
process perhaps). This is a convenient way of dealing with firewalls.

\section{Configuring .emacs}

Add the following line to yout {\tt .emacs} file:
\begin{verbatim}
(setq ange-ftp-ftp-program-name "nftp.pl")
\end{verbatim}

That should be it!

\section{Things that can go wrong}

nftp.pl deliberately suppresses error messages to avoid confusing
ange-ftp. Try testing a file transfer using something like the
following command:
\begin{verbatim}
scp -q -P 22 username@remote.host/.profile /tmp
\end{verbatim}
Any error messages you recieve should be taken seriously. For example,
earlier versions of scp do not support -q, or your default PATH on the
remote system may not include scp.

\end{document}
